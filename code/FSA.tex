\documentclass[12pt]{article}
\usepackage{longtable}
\usepackage{geometry}
%\geometry{left=1.5 in,right=1.5 in,top=1.2in,bottom=1.2in}

\usepackage{natbib}
\setcitestyle{authoryear}  % set citation style to authoryear
\bibliographystyle{plainnat} % use the plainnat instead of plain

\usepackage{booktabs}
\usepackage{amssymb, amsmath, amsfonts}
\usepackage{outlines}
\usepackage{soul}
%\usepackage{toda}
\usepackage[gen]{eurosym}
\usepackage{graphicx}
\usepackage{fancyhdr}
\usepackage{authblk}
\usepackage{lineno}
\usepackage[hyphens]{url}
\usepackage{hyperref}
\hypersetup{
 colorlinks = true, %Colours links instead of ugly boxes
 urlcolor  = blue, %Colour for external hyperlinks
 linkcolor = blue, %Colour of internal links
 citecolor = blue %Colour of citations
}
\pagestyle{fancy}
\usepackage{outlines}
\usepackage{caption}
%\captionsetup[table]{name=Figure}
\graphicspath{{../input/}}
\usepackage[outdir=./]{epstopdf}


\usepackage{enumitem}
\setlist[enumerate,2]{label=\roman*)}
\setlist[enumerate,3]{label=\Roman*)}
\setlist[enumerate,4]{label=\roman*)}

%\renewcommand{\familydefault}{\sfdefault}
% ------------------------------- use \citep{FAQs} to cite
\usepackage{filecontents}
\begin{filecontents}{\jobname.bib}
}

@article{Ross2016a,
  title   = {{The Supervisory Capital Assessment Program}},
  author  = "Ross, Chase P.",
  year   = "2016",
  doi    = " ",
  volume  = " ",
  journal  = "Yale Program on Financial Stability Intervention Case",
  issn   = " ",
 note={\newline\url{http://papers.ssrn.com/sol3/papers.cfm?abstract_id=2722712}
}}

@article{Ross2016b,
  title   = {{2010 CEBS EU-Wide Stress Test}},
  author  = "Ross, Chase P.",
  year   = "2016",
  doi    = " ",
  volume  = " ",
  journal  = "Yale Program on Financial Stability Intervention Case",
  issn   = " ",
 note={\newline\url{http://som.yale.edu/download-ypfs/2010-CEBS-EU-Wide-Stress-Test.pdf}
}}

@article{SCAPResults,
  title   = {{The Supervisory Capital Assessment Program: Overview of Results}},
  journal = "Board of Governors of the Federal Reserve System",
  author  = {{Federal Reserve}},
  year   = "2009",
 note={\newline\url{http://www.federalreserve.gov/newsevents/press/bcreg/bcreg20090507a1.pdf}
}}


@article{BRSAnnouncement,
  title   = {{Financial support to the banking industry}},
  author  = {{HM Treasury}},
  year   = "2008",
 note={\newline\url{http://webarchive.nationalarchives.gov.uk/20090224112406/http://www.hm-treasury.gov.uk/press_100_08.htm}
}}

@article{Nov2008,
  title   = {{FSA Statement on Capital Approach Utilised in UK Bank Recapitalisation Package}},
  author  = {{Financial Services Authority}},
  year   = "2008",
 note={\newline\url{http://webarchive.nationalarchives.gov.uk/20091212045501/http://www.fsa.gov.uk/pages/Library/Communication/Statements/2008/capapp.shtml}
}}

@article{Jan2009,
  title   = {{FSA Statement on Regulatory Approach to Bank Capital}},
  author  = {{Financial Services Authority}},
  year   = "2009",
 note={\newline\url{http://webarchive.nationalarchives.gov.uk/20091212045501/http://www.fsa.gov.uk/pages/Library/Communication/Statements/2009/bank_capital_.shtml}
}}

@article{Murphy,
  title   = {{HM Treasury Refuses FOI Stress Test Data}},
  journal = "Financial Times",
  author  = "Murphy, Paul",
  year   = "2009",
 note={\newline\url{http://ftalphaville.ft.com/2009/05/22/56163/hm-treasury-refuses-foi-stress-test-data/}
}}

@article{Barclays1,
  title   = {{Barclays a step closer to passing stress test}},
  journal = "Financial Times",
  author  = "Croft, Jane",
  year   = "2009",
 note={\newline\url{http://on.ft.com/23pwjs2}
}}

@article{Lex,
  title   = {{Barclays}},
  journal = "Financial Times",
  author  = "Lex",
  year   = "2009",
  doi    = " ",
  volume  = " ",
  journal  = " ",
  issn   = " ",
 note={\newline\url{http://on.ft.com/23pyfkb}
}}

@article{Results,
  title   = {{FSA Statement on the Use of Stress Tests}},
  author  = {{Financial Services Authority}},
  year   = "2009",
 note={\newline\url{http://www.fsa.gov.uk/pages/Library/Communication/PR/2009/068.shtml}}
}}

@article{Stressy,
  title   = {{FSA releases `not-so-stressy'-test methodology}},
  journal = "Financial Times",
  author  = "Kaminska, Isabella",
  year   = "2009",
 note={\newline\url{http://on.ft.com/1g2ZxXI}
}}

@article{BoE2015,
  title   = {{Financial Stability Report: Resilience of the UK Financial System}},
  author  = {{Bank of England}},
  year   = "2015",
  journal  = "Financial Stability Report",
 note={\newline\url{http://www.bankofengland.co.uk/publications/Documents/fsr/2015/fsr37sec7.pdf}}
}

 @misc{Dunfermline,
    title   = {{Dunfermline Building Society Property Transfer Instrument 2009}},
    author  = {{Bank of England}},
    year   = "2009",
 note={\newline\url{http://www.bankofengland.co.uk/financialstability/Documents/role/risk_reduction/srr/resolutions/dunfermlinecombinedtransferinstrument.pdf}
 }}

 @misc{BankingAct,
   title   = {{2009 Banking Act}},
   author  = {{Parliament}},
   year   = "2009",
   note={\newline\url{http://www.legislation.gov.uk/ukpga/2009/1/introduction},
}}

   @article{McKnight,
     title   = {{UK: The Banking Act 2009}},
     journal = "Mondaq",
     author  = "McKnight, Andrew",
     year   = "2009",
     note={\newline\url{http://www.mondaq.com/x/80880/Credit+Crisis+Emergency+Economic+Stabilization+Act/The+Banking+Act+2009}
}}

   @misc{DunferlminePress,
     title   = {{Dunfermline Building Society}},
  author  = {{Bank of England}},
     year   = "2009",
     note={\newline\url{http://www.bankofengland.co.uk/archive/Documents/historicpubs/news/2009/030.pdf}
       }}

@article{FSAReport,
     title   = {{FSA Annual Report 2009/2010}},
     author = "Financial Services Authority",
     year   = "2010",
     note={\newline\url{{http://www.fsa.gov.uk/pubs/annual/ar09_10/ar09_10.pdf}}
       }}
@article{Haldane,
  title   = {{Why Banks Failed the Stress Test}},
  journal = "Bank of England",
  author  = "Haldane, Andrew G.",
  year   = "2009",
 note={\newline\url{http://www.bis.org/review/r090219d.pdf}}
}

\end{filecontents}


\begin{document}

\lhead{}
\rhead{}

\renewcommand{\headrulewidth}{0.0pt}
\renewcommand{\footrulewidth}{0.0pt}

%% DELETE ABOVE TO GO BACK TO DEFAULT %%%%%%

\title{Financial Services Authority 2008/9 Stress Test}%\thanks{This case is one of a series of intervention studies.}}
%\author{Chase Ross\thanks{Yale Program on Financial Stability, Yale School of Management \texttt{\href{mailto:chase.ross@yale.edu}{chase.ross@yale.edu}}}}
\author{Chase P. Ross\thanks{\texttt{\href{mailto:chase.ross@yale.edu}{chase.ross@yale.edu}} \\ This paper is one in a series on stress tests conducted in 2009 and 2010. See \citet{Ross2016a} for a discussion of the US 2009 Stress Test and \citet{Ross2016b} for a discussion of the 2010 CEBS EU-wide stress test.}}
\affil{Yale Program on Financial Stability \\ Yale School of Management}
%\date{January 14, 2016}



\maketitle

\begin{abstract}

In response to deteriorating market conditions UK financial regulators announced the Bank Recapitalization Scheme ``to ensure the stability of the financial system and to protect ordinary savers, depositors, businesses and borrowers'' in October 2008. The program used a three-pronged approach to address liquidity needs of banks, provide guarantees on certain bank debt, and recapitalize banks should it prove necessary. For this, the Financial Services Authority (FSA) conducted stress a test of the involved institutions to determine appropriate capital buffers for each bank. The FSA stress test is notable as it was the first ``wartime'' stress test conducted during the Global Financial Crisis, and was an important predecessor for later stress tests conducted in the US (in May 2009) and the EU (in October 2009 and July 2010). However, the test was also notably different with regards to the level of public disclosure. The test was ultimately, in conjunction with various other concurrent programs, ultimately helped ease pressure on the UK banking system.
\newline
\newline
\textbf{JEL Classification}: G01, G28, G20, H12, H81
\newline
\textbf{Keywords}: crisis management stress test, crisis intervention, Financial Services Authority, capital requirements, systemic risk

\end{abstract}
\newpage
\tableofcontents
\newpage

\section{Overview}

\subsection{Background}

On October 8, 2008, HM treasury announced the UK Tripartite's comprehensive plan to support the banking system, the so-called Bank Recapitalisation Scheme. The UK Tripartite authorities include HM Treasury, the Financial Services Authority (FSA), and the Bank of England (BoE). The Tripartite's announcement came amid significant market turmoil, with the cost of protection on the largest UK banks nearly tripling over a few week period in late September after the collapse of Lehman Brothers, and the FTSE 100 had fallend roughly 40\% from the year prior. Figure~\ref{timeline} charts the relevant market indicators along with key dates in 2008 and 2009.

\begin{figure}[h]
\caption{Market Turmoil through 2008 and 2009}\label{timeline}
\centering
\includegraphics[width=\textwidth]{timeline.pdf}
\raggedright
\textit{\footnotesize Source: Bloomberg.}
\end{figure}

The Tripartite's plan featured three main components:\footnote{The Yale Program on Financial Stability has produced similarly structured papers on each of the three components of the Bank Recapitalization Scheme. See [tk].}

\begin{enumerate}
\item Provision of at least \pounds 200 billion of liquidity through an expansion of the Special Liquidity Scheme,
\item Boost Tier 1 capital at key financial institutions, with \pounds 25 billion available immediately and an incremental \pounds 25 billion available as further support ``at the request of an eligible institution,'' and
\item Creation of a guarantee program of \pounds 250 billion for new wholesale debt from banks with a plan to boost Tier 1 capital in the amount supervisors deem appropriate.
\end{enumerate}

The plan's announcement identified eight firms which had confirmed their participation in the plan: Abbey (owned by Santander), Barclays, HBOS, HSBC Bank plc, Lloyds TSB, Nationwide Building Society, Royal Bank of Scotland, and Standard Chartered. The plan received mixed response from market participants, and on the same day the BoE, Federal Reserve, ECB and central banks of Sweden, Canada and the UAE all jointly cut rates by 50 basis points. CDS spread tightened across the board but equities fell, especially as Barclays had announced it did not need any additional capital just weeks before. \citep{BRSAnnouncement}. Five days later, on October 13, the UK government took a 63\% stake in RBS and a 44\% stake in Lloyds as part of the recapitalization plan.

\subsection{Program Description}

\paragraph{Purpose}

The FSA used stress testing to assist other Tripartite members as they made ``policy decisions such as access to the Credit Guarantee Scheme and the Asset Protection Scheme.'' Additionally, the FSA parsed its 2009 stress test in the context of its routine use as an ``integral element of [their] ongoing supervisory approach'' as well as a component of the 2009 CEBS EU-wide stress test. That is, the test was not a one-off exercise, but rather embedded into the FSA routine supervisory practices. \citep{Results}.

\paragraph{Legal Authorization}

The Banking Act 2009, effective February, 2009, created a ``Special Resolution Regime'' which gave regulators permanent tools to deal with financial institutions in distress. \citep{BankingAct}. The Banking Act provides a path through which regulators can place the distressed institution into the Bank Insolvency Procedure, transfer a portion or all of the institution to a private purchaser or bridge bank owned by the Bank of England, or transfer the institution to ``temporary public ownership by the Treasury (TPO).'' \citep{DunfermlinePress}.\footnote{Banking Act 2009 \S \S 11(2),12(2),34,35,36,37,38,39,40,84. More broadly, \citet{Dunfermline} is an example of the implementation of this process. } Before the stabilization measures provided in the Act may begin, the FSA must first, together with the Treasury and the Bank of England, must be satisfied with the following conditions:\footnote{Banking Act 2009 \S 7}

\begin{enumerate}
\item ``the bank has failed, or is likely to fail, to satisfy the threshold conditions for authorisation under \S 41(1) of and Schedule 6 of the Financial Servicse and Markets Act 2000 (e.g. as to adequate resources to carry on its regulated activities,'' and
\item ``that it is unlikely this will be rectified, ignoring the use of stabilisation power or financial assistance provided by the Treasury or the Bank of England (other than ordinary market assistance offered by the latter on its usual terms)...'' \citep{BankingAct} and \citep{McKnight}.
\end{enumerate}

The Bank of England must be satisfied that exercise of the stabilization power is in the public interest with regards to the stability of the financial system of the UK, the maintenance of public confidence in the stability of the banking system, or to depositor protection. Finally, Treasury must recommend the Bank of England use its stabilization power, and provide consent to the use of public funds.\footnote{Banking Act 2009 \S \S 8, 78}

\paragraph{Coverage}

The FSA did not release a list of stressed banks, although in subsequent publications the FSA said it tested all major banks in the UK and focused on Lloyds/HBOS and RBS. The test also included certain building societies.The \textit{Financial Times} reported the FSA also tested, and passed, Barclays conditional on its sale of a certain business line. Moreover, the FSA emphasized that its use of stress testing was a component of its ongoing supervisory approach across all firms it regulates. \citep{Results} and \citep{Barclays1}.

\paragraph{Scenarios} The test examined institutions' health over the next five years, but centered its focus on the first three years. The FSA stress scenario tested financial institutions' ability to withstand a recession more severe than any since World War Two. This recession scenario included fall of peak-to-trough GDP of over 6\%, with growth returning in 2011 to trend growth in 2012. The scenario also provided certain interest rate assumption, but these were not publicly released. The scenario included an increase in unemployment to 12\%, a fall in house prices of 50\% peak-to-trough and a fall in commercial real estate of 60\% peak-to-trough. The test examined revenue projections given this macro backdrop by using the probability of default (PD) and the loss given default (LGD) across an institution's loan book along with declining market prices for instruments held in the trading book.\citep{Results}. The FSA did examine possible management action during times of stress and the test allowed the balance sheet to vary throughout the testing horizon either via capital raising or asset sales.\footnote{The 2009 and 2010 EU-Wide stress tests did not have such an assumption; these tests assumed that the balance sheet would remain constant throughout the testing horizon.}

When the FSA published details on its testing methodology in late May 2009, the test's scenario surprised some observers. Where rumors had expected a scenario with a 50\% fall in house prices and a 16\% decline in peak-to-trough GDP, the test actually included a 50\% fall in house prices and a fall of 6\%. ``If the wider market presumption was indeed a 16 percent GDP drop, the much less stressy scenario revealed could certainly explain some of the falls seen in bank share prices at the open...'' \citep{Stressy}.

\paragraph{Hurdle rate}

The FSA expected banks to maintain Core Tier 1 of at least 4\% of risk-weighted assets and Tier 1 Capital of at least 8\%; building societies which applied for the Credit Guarantee Scheme to maintain a stressed Tier 1 capital ratio of 7\%. \citep{Results} The FSA also changed the risk weighting calculation from a point-in-time method to a through-the-cycle, with the effect of ``significantly reduc[ing] the requirement for additional capital resulting from the procyclical effect.'' \citep{Jan2009}.

\subsection{Outcomes}

The FSA did not publish results at any level (and refused to do so in the future) so direct measure of the stress test's impact is not possible. However, in subsequent publications the FSA noted that it focused particularly on the stress tests of Lloyds and RBS. The FSA found both to be insufficiently capitalized in case of a deterioration in the macro environment. Both Lloyds and RBS were unable to raise private capital initially, however as market conditions improved in through the second half of 2009 only RBS ultimately required assistance from the APS in November 2009. Lloyds ultimately raised private capital but was required to pay a fee ``to the taxpayer for the implicit protection provided.'' \citep{FSAReport}.

Despite outright disclosures, other effects of the FSA's stress test are visible. Barclays appeared to pass the stress test with sufficient capital conditional on its sale of its iShares business. \citep{Barclays1}. A second example: on March 28 the FSA determined the Dunfermline Building Society ``was likely to fail to meet the FSA threshold conditions for authorisation and that there was no option available which would have enabled the company to satisfy the threshold conditions.'' \citep{DunferlminePress}. HM Treasury noted that had regulators not acted Dunfermline would have been unable to meet depositors' claims. UK regulators used the Special Resolution Regime process provided by the 2009 Banking Act to transfer Dunfermline's deposit business to Nationwide, its social housing loans to a bridge bank controlled by the Bank of England named the DBS Bridge Band Ltd, and the remainder of Dunferlmine's businesses -- including commercial loans, acquired residential mortgagse, subordinated debt and most treasury assets -- into the Building Society Special Administration Procedures (BSSP). \citep{DunferlminePress}.

\section{Key Design Decisions}\label{keydesign}

\subsection{The FSA did not initially disclose details of the test and never disclosed bank-by-bank results.}

A key aspect to the FSA's stress test was the evolution of disclosure from the time the Bank Recapitalisation Scheme's announcement in October 2008 to the release of relatively more specific design information in May 2009. The plan as announced in October did not explicitly mention the FSA nor the stress test. Rather, the FSA published a press release in November 2008 which provided just three paragraphs of details on the test's method and aims, essentially declining to publicly describe the specifics of the test or when the results would be announced (if at all). The relevant paragraphs in their entirety read:

\begin{quote}
  Within the Tripartite structure, the FSA was responsible, in consultation with the other two authorities, for determining the appropriate level of capital for each individual institution. In reaching this determination two factors were taken into account:

  \begin{enumerate}
    \item Ensuring that the amount of capital for each institution would sustain confidence in that institution.
    \item Ensuring that each individual institution would have a sufficient capital buffer over minimum capital requirements both to absorb losses that might ensue from a recession and to continue lending on normal commercial criteria.
  \end{enumerate}

  To ensure broad consistency between different institutions, the process included utilisation of a stress test based on some standard assumptions but with weightings tailored to the specific institutions. The FSA used as common benchmarks within this framework ratios of capital to risk weighted assets of total Tier 1 Capital of at least 8\% and Core Tier 1 Capital, as defined by the FSA, of at least 4\% after the stressed scenario.

\end{quote}

The FSA released no comments on the test design or results, save a brief note on changes to the additional capital held as a result of Basel's procyclical effect, until May 28, 2009. Beginning in February 2009, US financial regulators had begun a stress test of the major US bank holding companies called the Supervisory Capital Assessment Program (SCAP). In April 2009 the Federal Reserve publicly published the methodology and scenarios used in the stress test, and on May 7, 2009 US regulators published bank-by-bank details describing exposures by asset class along with estimates of bank-specific capital shortfalls. It was dramatic departure from the disclosure precedent by the FSA.

In the period between formal announcement of the stress test and May 2009, the FSA provided little information on the stress test. After Lloyds and RBS were recapitalized with public funds in October, attention turned to Barclays as the last large bank which might require substantial public support. However, this turned out to not be the case when the \textit{Financial Times} learned that the FSA had concluded its stress test of Barclays in April 2009 with no additional capital injection. \citep{Barclays1} and \citep{Lex}.

In light of the opaque announcement regarding Barclays, some market participants felt that SCAP-like disclosures would bolster both confidence in the banking system but also in the FSA:

\begin{quote}
  More than ever, however, the FSA’s credibility is on the line. By giving the bank the all-clear, it has indicated that it is satisfied the group will be able to maintain core tier one capital of 4 per cent at the trough of the recession, the minimum requirement it stipulated in January... The FSA must address concerns over the rigour of its stress testing by detailing its worst-case assumptions. Is it sufficient to test Barclays against merely a replay of an early 1980s-type fall in employment and gross domestic product, for example? And does the FSA recognise the extent to which UK house prices remain hideously overvalued? A pass is worth little if the examiners are complicit in rampant grade inflation. \citep{Lex}.
\end{quote}

Pressure built for the FSA to release additional details after the May 7 SCAP disclosures in the United States, and Bloomberg submitted a Freedom of Information request to HM Treasury for the results of the stress test. HM Treasury denied the request, saying disclosure of the results ``at this time may lead to uncertainty in financial markets, either in relation to specific institutions or more generally... [s]uch instability could require further action by the authorities.'' UK regulators were likely reluctant to release the details of the stress test because it would be difficult to explain the differences between the scenarios used in the SCAP and the FSA tests, and also because markets may mistake the scenarios as official forecasts -- a particularly scary scenario given that rumors suggested the FSA was testing with a 50\% decline in home prices and a 16\% fall in GDP.\footnote{GDP ultimately dropped 4.2\% in the US peak to trough; 4.9\% in the UK.} \citep{Murphy}.

On May 28, the FSA released a detailed methodology statement, although the disclosures did not match the SCAP's level of detail. The release directly noted the impact SCAP had on the FSA's decision to release results, ``the publication in the US of the results of bank stress tests [provoked] substantial interest in the use of stress testing in other countries'' and that the two programs were not comparable: ``The UK authorities have not applied stress testing in the same way as in the US.'' \citep{Results}.

\subsection{The stress test was not a one-off program.}

The announcement emphasized the FSA's use of stress testing as a part of its ongoing supervisory regime. Rather than use the test as a one-off program, the FSA outlined its use of stress testing in its ongoing regulatory approach:

\begin{enumerate}
  \item ``Greatly increased the use of stress tests as an integral element of our ongoing supervisory approach.''
  \item ``Begun the process of embedding this revised approach in our intensive supervisory regime.''
  \item ``Used stress tests to inform policy decisions such as access to the Credit Guarantee Scheme (CGS) and the Asset Protection Scheme (APS) working closely with the other Tripartite authorities.'' \citep{Results}.
\end{enumerate}

This approach contrasted with contemporaneous tests in the US and EU which were both publicized as a one-off test. Accordingly, the FSA note ``[s]tress testing can and has been used in a variety of different ways, and the appropriate degree of disclosure varies according to the purposes of the tests\dots Since the FSA’s use of stress tests has not been a one-off exercise, but instead embedded in our regular supervisory processes, the FSA will not, as a matter of practice, be publishing details of the stress test results.'' \citep{Results}.

\section{Evaluation}

There is little evaluation of the FSA's 2009 stress test because the test was not publicized, its results were not released, and the FSA used the stress test as part of its ongoing supervisory approach. However, Andrew Haldane -- the head of financial stability at the Bank of England in 2009 -- described the shortcomings of the stress test that had been embedded into the normal regulatory process. \citep{Haldane}. He identifies three reasons that the stress test failed in 2008 and 2009: underestimating the likelihood of extreme events, poor understanding of spillover and linkages, and misaligned incentives. To combat these points he suggests five policies.

\paragraph{``Setting the stress scenario.''} It is important that regulators, not firms themselves, create the stress scenario as to avoid ``disaster myopia.'' Further, the test should be extreme enough that it represents a tail event.

\paragraph{``Regular evaluation of common stress scenarios.''} Basel II required banks to conduct annual stress tests, and financial institutions struggled to implement stress tests over a shorter time-frame. Regular testing would also allow direct comparisons across firms and encourage management to use the results of these tests in their normal decision process.

\paragraph{``Assessment of the second-round effects of stress.''} The results of the tests should be compared with second order effects like liquidity hoarding and fire-sales, and thereby encourage firms to think about spillover and contagions that comes with their own actions.

\paragraph{``Translation of results into firms' liquidity and capital planning.''} Results of stress tests, including the second-round effects, need to be used in management decisions and should be taken to firms' risk committees.

\paragraph{``Transparency to regulators and financial markets.''} Public disclosure of bank-specific results would impose some market discipline over management decisions. These disclosures should be standardized so that direct comparisons are possible, and the disclosures should be regularly published -- not just during moments of particular stress.


In its 2009/2010 annual report, the FSA summarized the impact of its stress test:

\begin{quote}
[The FSA] set a challenging stress test for banks based on a much more severe recession than the Bank of England’s central projection. This proactive approach resulted in banks significantly improving their capital positions which enabled them to better withstand the downturn. \citep{FSAReport}.
\end{quote}



\newpage
\phantomsection

\addcontentsline{toc}{section}{References}

\nocite{*}
\bibliography{\jobname}

\newpage


\section{Appendix A - List of Resources}

\subsection{Summary of Program}

\begin{itemize}
\item
\ul{FSA Statement on Capital Approach Utilised in UK Bank Recapitalisation Package}, Financial Services Authority, November 14, 2008 -- \emph{FSA press release providing its initial -- and less detailed -- description of the stress test.} \url{http://webarchive.nationalarchives.gov.uk/20091212045501/http://www.fsa.gov.uk/pages/Library/Communication/Statements/2008/capapp.shtml}
\item
\ul{FSA statement on its use of stress tests}, Financial Services Authority, May 28, 2009 -- \emph{FSA press release providing its most detailed description of the stress test.} \url{http://www.fsa.gov.uk/pages/Library/Communication/PR/2009/068.shtml}
\end{itemize}

\subsection{Implementation Documents}
\begin{itemize}
\item
\ul{Dunfermline Building Society Property Transfer Instrument 2009 }, Bank of England, March 30, 2009 -- \emph{Term sheet for the Dunfermline transaction in accordance with the ``Special Resolution Regime'' from the 2009 Banking Act.} \url{http://www.bankofengland.co.uk/financialstability/Documents/role/risk_reduction/srr/resolutions/dunfermlinecombinedtransferinstrument.pdf}
\end{itemize}

\subsection{Legal/Regulatory Guidance}

\begin{itemize}
\item
\ul{2009 Banking Act}, Parliament, February 2009 -- \emph{Act of Parliament authorizing the FSA, HM Treasury and Bank of England to use the ``Special Resolution Regime.''} \url{http://www.legislation.gov.uk/ukpga/2009/1/introduction}
\end{itemize}

\subsection{Press Releases/Announcements}

\begin{itemize}
\item
\ul{Financial support to the banking industry}, HM Treasury, October 8, 2009 -- \emph{Press release which
 announces the UK's major response to the banking crisis.} \url{http://webarchive.nationalarchives.gov.uk/20090224112406/http://www.hm-treasury.gov.uk/press_100_08.htm}
\item
\ul{FSA Statement on Capital Approach Utilised in UK Bank Recapitalisation Package}, Financial Services Authority, November 14, 2008 -- \emph{FSA press release providing its initial -- and less detailed -- description of the stress test.} \url{http://webarchive.nationalarchives.gov.uk/20091212045501/http://www.fsa.gov.uk/pages/Library/Communication/Statements/2008/capapp.shtml}
\item
\ul{FSA Statement on regulatory approach to bank capital}, Financial Services Authority, January 19, 2009-- \emph{Press release
 providing additional details on the FSA stress test and hurdle rates.} \url{http://webarchive.nationalarchives.gov.uk/20091212045501/http://www.fsa.gov.uk/pages/Library/Communication/Statements/2009/bank_capital_.shtml}
\item
\ul{FSA statement on its use of stress tests}, Financial Services Authority, May 28, 2009 -- \emph{FSA press release providing its most detailed description of the stress test.} \url{http://www.fsa.gov.uk/pages/Library/Communication/PR/2009/068.shtml}
\end{itemize}

\subsection{Media Stories}

\begin{itemize}
\item
\ul{Barclays a step closer to passing stress test}, Jane Croft, Financial Times, March 26, 2009 -- \emph{Article providing details on the timing of Barclays' stress test and its likelihood of passing.} \url{http://www.ft.com/intl/cms/s/0/76fedcf8-1a47-11de-9f91-0000779fd2ac.html}
\item
\ul{Barclays}, Lex, Financial Times, March 27, 2009 -- \emph{Article announcing that Barclays had passed the stress test.} \url{http://www.ft.com/intl/cms/s/3/d170d4fe-1adb-11de-8aa3-0000779fd2ac.html}
\item
\ul{HM Treasury refuses FOI stress test data}, Paul Murphy, Financial Times, May 22, 2009 -- \emph{Article describing the low level of disclosure around the FSA's stress tests and HM Treasury's rejection of Freedom of Information requests about the tests.} \url{http://www.ft.com/intl/cms/s/3/d170d4fe-1adb-11de-8aa3-0000779fd2ac.html}
\item
\ul{FSA releases ‘not-so-stressy’-test methodology}, Isabella Kaminska, Financial Times, May 28, 2009 -- \emph{Article describing the scenarios used in the FSA's stress tests.} \url{http://ftalphaville.ft.com/2009/05/28/56325/fsa-releases-not-so-stressy-test-methodology/}
\end{itemize}

\subsection{Key Academic Papers}

\begin{itemize}
\item
\ul{Why Banks Failed the Stress Test},
 Andrew G. Haldane, February 10, 2009 -- \emph{Speech identifying the problems with routine stress tests and proposals to fix them.} \url{http://www.bis.org/review/r090219d.pdf}
\end{itemize}

\subsection{Reports/Assessments}

\begin{itemize}
\item
\ul{UK: The Banking Act 2009},
Andrew McKnight, June 8, 2009-- \emph{Report discussing the impact of the 2009 Banking Act.} \url{http://www.mondaq.com/x/80880/Credit+Crisis+Emergency+Economic+Stabilization+Act/The+Banking+Act+2009}
\item
\ul{FSA Annual Report}, Financial Services Authority, June 2010 -- \emph{FSA annual report discussing the stress test conducted during the crisis and the impacts of those tests.} \url{http://www.fsa.gov.uk/pubs/annual/ar09_10/ar09_10.pdf}
\end{itemize}

\section{Appendix B - Road Map}

The following is a list of the key design decisions that will likely have to be made in implementing a program similar to the 2008/9 FSA stress test, a' program intended to assess the capital needs of financial institutions receiving taxpayer support during a period of heightened uncertainty around potential losses in the banking system.

\subsection{Key Questions}

\begin{outline}[enumerate]

\1 Which agency or agencies have the authority and expertise to conduct the stress test?
\2 What is the basis of this authority?
\2 What particular elements/terms must be satisfied to fit within the authority?
\2 After designing, have all required elements been satisfied?
\1 What, if any, capital backstop should be available to firms which undergo the stress test?
\2 Does the existence or lack of a public capital backstop affect market views of the test's credibility?
\2 Is any additional authority required in order to provide a capital backstop?
\2 How can the backstop be structured to compel firms to first raise private capital and use the public capital as a less preferred option?
\2 How long should firms be allowed to seek private capital before turning to the public backstop?
\1 How should a public capital backstop be structured?
\2 What sort of security should the public capital be provided through?
\2 Should economic conditions worsen, can the public capital convert into common equity (at a discount)?
\2 What other constraints will firms using public capital face? (E.g. executive compensation caps, restrictions on common stock dividends, buybacks and cash acquisitions, etc.)
\1 Which firms are included in the stress test?
\2 How many firms can be credibly tested given the testing agency's resources?
\1 How transparent should the test results be? What level of granularity for estimates should be publicly available?
\2 If the issue is sovereign exposures, how should these sovereign exposures be stressed?
\1 How can the regulators ensure the test is viewed as credible?
\2 How should existing public support be incorporated into the test?
\2 What metric or measure should regulators target to assess capital adequacy?
\3 What is the target hurdle rate?
\3 What data on bank holdings and capital adequacy does the testing agency collect as part of its regular bank examination process? Is this data sufficiently granular for the test or will further data need to be collected?
\3 Should the test focus on Tier 1 capital, Tier 1 Common capital, Core Tier 1 capital, tangible common equity, a combination of these or something else?
\4 Should hybrid securities associated with pubic support be included?
\4 For example, should preferred equity, goodwill and intangible assets be included in the equity component?
\4 Should the denominator be based on risk-weighted assets, tangible assets or something else?
\1 Over what time frame is the stress test examining capital adequacy?
\2 Should the stress test measure in-the-moment or measure ``in the stress''?
\1 What economic scenarios are used to stress the firms?
\2 How do you ensure consistency in the forecast parameters across many jurisdictions?
\2 Who produces the scenarios, and at what level of geographic specificity are they produced?
\2 How many economic scenarios are used to stress the firms?
\2 How do these scenarios compare with contemporary private forecasts?

\end{outline}

\subsection{Implementation Steps}

\begin{enumerate}

\item Develop the description of the test, including legal authority, purpose, firm eligibility, a general timeline, et cetera and seek input from industry and other stakeholders.
\item If necessary, seek approval for the program, funding et cetera.
\item If a capital backstop will be used, establish a description of the backstop with legal authority, firms eligible to receive the backstop, the mechanics of the capital injection, et cetera and seek input from industry and other stakeholders. Produce term sheet for the program.
\item Produce economic scenarios with which to stress capital adequacy and distribute it to the tested firms.
\item If firm-level data at a sufficiently granular level is not available from the traditional bank examination processes, collect the relevant data from firms.
\item Draft detailed FAQs and template for published results.
\item Using the provided economic scenarios and bank portfolio data, both supervisors and firms individually produce capital adequacy estimates.
\item Compare supervisors' capital adequacy estimates with firms' own estimates and reconcile differences.
\item If necessary, develop instructions for completing the documentation necessary to participate in the capital back stop.
\item Publicly release stress test results.

\end{enumerate}


\end{document}
